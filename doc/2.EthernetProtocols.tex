\section{Ethernet protocols}

Ethernet protocols refer to a family of computer networking technologies
used in Local Area Network. Developed by Xerox in 1973, it have been
commercially introduced in 1980. Since its release, it has retained a
good degree of compatibility.

With the time, Ethernet has been standardized in IEEE 802.3 and has
replaced most of the other wired LAN technologies. Some of this features
such as 48-bit MAC address and Ethernet frame format have influenced
other networking protocols.

The Ethernet system consists of three basic elements:

\begin{enumerate}
    \item The physical medium used to carry Ethernet signal between
    computers.
    \item A set of medium access control rules embedded in each Ethernet
    interface that allow multiple computers to fairly arbitrate access
    to the shared Ethernet channel
    \item An  Ethernet frame that consists of a standardized set of bits
    used to carry data over the system.
\end{enumerate}

Each host that is using Ethernet operates independently of all other
station on the network, there is no central controller. All hosts are
connected the same link (the medium). Therefor before sending data
on this link, they need to listen the channel to make sure nobody else
is using it. The medium access control mechanism used by Ethernet use a
system called Carrier Sense Multiple Access with Collision Detection
(CSMA/CD).

As Ethernet is a widely used standard and it design robustness has been
proved and influenced many other protocols, it appear to be very
interesting to study this subject and to try to simulate this protocol.
We will moreover do some performance evaluation on the Ethernet
protocols to measure and compare some parameters like delay, delay
jitter and throughput.

