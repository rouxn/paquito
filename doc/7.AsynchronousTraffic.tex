\section{Asynchronous traffic}

TODO: Introduce asynchronous traffic.

\begin{description}
    \item Arrival process : $ \frac{e^{b} \cdot b^{k}}{k!}$ (Poisson distribution)
    \item Traffic load = packet arrival time rate  (Poisson distribution : $ \frac{e^{b} \cdot b^{k}}{k!}$) $\times$ packet size (Exponential distribution : $\lambda \cdot e^{ \lambda \cdot t}$)

	\begin{description}  
        \item b = $\lambda \cdot$t
        \item t is used to define the interval 0 to t   
        \item $\lambda$ is the total average arrival rate in packets
        \item k is the total number of packets in the interval 0 to t
   	\end{description}

    \item Average packet delay = $\overline{x} = \dfrac{1}{n} \sum_{i=1}^{n} x_{i}$
    \item Delay Variance = $ V(X) = \dfrac{1}{n} \sum_{i=1}^{n} x_{i}^{2} - \overline{x}$

    	\begin{description}
            \item X : Delay
            \item n = number of packets
            \item $x_{i} = delay for packet number \times i$
            \item $\overline{x} = average packet delay$
       	\end{description}

    \item Thoughput = $\frac{number of byte sent}{total time}$
\end{description}

\subsection{Poisson process}

Contrary to the synchronous traffic, in the asynchronous traffic the number of packet received vary with time. 

In the case of asynchronous traffic, the reception of the packet is directed dictated by the poisson process  $ \frac{e^{\lambda} \cdot \lambda^{k}}{k!}$ . The number of packet we will receive depends on the number of packet we had already received. Let k represent the number of packet we want to send. At $ t=0 $, we have send and receive 0 packets. So if we decide to send at that moment, the number of packet received will be given by P(0). Consider $ P(0)=p $, with $ 0 < p < 1 $ and p is round at the higher figure. So at the first emission we only receive p\% packet. The total number of packet received will be $ k*p $. So we still must send $ (1-p)*k $ packets. At the second emission we will receive, according the same principle, P(p) packets, with $ 0 < P(p) < p $. Now the receiver has become $ (p+P(p))*k $ packets. He still needs $ (1-p+P(p)))*k $ packets. More the time passed, more the number of packet send decrease until we got a number near 0. In that case, we have sent all the packet.


