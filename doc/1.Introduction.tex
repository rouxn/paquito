\section{Introduction}

The network \& protocol course that we attempt teach us foundations and
advanced aspects of networking such as data link network and MAC, packet
switching, internetworking, end-to-end protocols and also congestion
control and resource allocation.

As a part of this course, we need to do a team project allowing us to
apply notions that we have studied. The goal of the project is to do
simulation of a Medium Access Control (MAC) protocol.

Moreover this simulation will allow us to evaluate the performances of
the used medium access control protocol.

The medium access control protocol is a sub-layer of the data link layer
(layer 2) specified in the seven-layer OSI model and the four layer
TCP/IP model (layer 1). It provides addressing and channel access
control mechanisms making communication between different hosts
possible within a multiple network.

The MAC protocols performs many function such as receiving and
transmitting frames, appending check sequence to frame, discarding
malformed frames or even backoff functions.

Many MAC protocols exists, such as Ethernet, Wifi or Token Ring. Each of
this protocols have it own multiple access protocols taking advantage of
the used medium constraints or topology. For example, the Ethernet
protocol use CSMA/CD, the Wifi protocol use CSMA/CA and the Token Ring
protocol use the Token bus.

For the purpose of this project, we focused on only one medium access
control protocol.

